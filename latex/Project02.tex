\documentclass[11pt, a4paper]{article}

\usepackage[T1]{fontenc}
\usepackage[utf8]{inputenc}
\usepackage[norsk]{babel}
%\usepackage{amsmath}
\usepackage{graphicx}
\usepackage{enumerate}
\usepackage{mathtools}
\usepackage{listings}
\usepackage{pdfpages}
\usepackage{tikz}
%\usepackage{multirow}
\usepackage{cite}
\usepackage{algorithm}
\usepackage{algpseudocode}
% \usepackage{hyperref}
\usepackage{url}
\usepackage{amssymb}

\setcounter{tocdepth}{1}

\lstset{language=C++, commentstyle=\textcolor[rgb]{0.00,0.50,0.00}, keepspaces=true, columns=flexible, basicstyle=\footnotesize, keywordstyle=\color{blue}, showstringspaces=false, inputencoding=ansinew}

\mathtoolsset{showonlyrefs}

\author{Eivind Brox}
\newcommand{\db}{\text{dB}}

\title{Prosjekt 1 FYS3150}
\pagenumbering{roman}
\setcounter{tocdepth}{3}
\setcounter{page}{0}
\date{\today}
\begin{document}
\maketitle
\thispagestyle{empty}
\clearpage

\pagestyle{headings}
\tableofcontents
\clearpage

\pagenumbering{arabic}
\section{Introduction}


\clearpage


\section{Method}
We first consider the solution to the radial part of Schr\"odinger's equation
\begin{gather}
  -\frac{\hbar^2}{2 m} \left ( \frac{1}{r^2} \frac{d}{dr} r^2
  \frac{d}{dr} - \frac{l (l + 1)}{r^2} \right )R(r) + V(r) R(r) = E R(r).
\end{gather}
Where $V(r) = (1/2)kr^2$ is the harmonic oscillator potential. $E$ is the harmonic oscillator energy in three dimensions, and $k=m\omega^2$. The frequency of the oscillator is $\omega$ and the energies are calculated as
\begin{gather}
E_{nl}=  \hbar \omega \left(2n+l+\frac{3}{2}\right)
\end{gather}
with $n,l = 0,1,2,\dots$

The quantum number $l$ is the orbital momentum of the electron, and as we are now in spherical coordinates we have that $r \in [0,\infty)$.  
%
Then we substitute $R(r) = (1/r) u(r)$ and obtain
%
\begin{gather}
  -\frac{\hbar^2}{2 m} \frac{d^2}{dr^2} u(r) 
       + \left ( V(r) + \frac{l (l + 1)}{r^2}\frac{\hbar^2}{2 m}
                                    \right ) u(r) = E u(r)
\end{gather}
We can introduce a dimensionless variable $\rho = (1/\alpha)r$, where $\alpha$ is a constant with dimension length. We will also be using $l=0$ in this project. Thus, we get
\begin{gather}
  -\frac{\hbar^2}{2 m \alpha^2} \frac{d^2}{d\rho^2} u(\rho) 
       + \frac{k}{2} \alpha^2\rho^2u(\rho)  = E u(\rho) .
\end{gather}
where we also have inserted 
\begin{gather}
V(\rho) = (k/2)\alpha^2\rho^2u(\rho)
\end{gather}
Multiplying with $2m\alpha^2/\hbar^2$ on both sides, we get
\begin{gather}
-\frac{d^2}{d\rho^2} u(\rho) 
       + \frac{mk}{\hbar^2} \alpha^4\rho^2u(\rho)  = \frac{2m\alpha^2}{\hbar^2}E u(\rho) .
\end{gather}
Now we fix $\alpha$ to
\begin{gather}
\alpha = \left(\frac{\hbar^2}{mk}\right)^\frac{1}{4}
\end{gather}
and define
\begin{gather}
\lambda = \frac{2m\alpha^2}{\hbar^2}
\end{gather}
to obtain the following version of Schr\"odinger's equation
\begin{gather}
-\frac{d^2}{d\rho^2}u(\rho) +\rho^2u(\rho) = \lambda Eu(\rho)
\end{gather}
This is the eigenvalue equation for which we are interested in calculating the eigenvalues numerically. For this simplified case we have the analytical solutions $\lambda_0 = 3$, $\lambda_1 = 7$, $\lambda_2 = 11,\dots$ We can utilize this to verifie our work.

To start off we use the following discretization of the second derivative
\begin{gather}
u''(\rho) = \frac{u(\rho+h)+u(\rho-h)-2u(\rho)}{h^2}+\mathcal{O}(h^2)
\end{gather}
where $h$ is the steplength.

We have that the boundary conditions are $u(0) = 0$ and $u(\infty) = \infty$. The problem is that we have no way to represent infinity on the computer. We thus have to spesify a value for $p_\text{max}$ which gives us a 'good enough' approximation for the number of eigenvalues we need to find, and the accuracy we want. $\rho_\text{min} = 0$ trivially.

Our step length then becomes 
\begin{gather}
h = \frac{\rho_\text{max}-\rho_\text{min}}{N} = \frac{\rho_\text{max}}{N}
\end{gather}
where $N+1$ is the number of points we are using for our discretization. 
We may then define a arbritrary value of $\rho_i$ as 
\begin{gather}
\rho_i = \rho_\text{min} + ih = ih
\end{gather}
for $i=0,1,2,\dots,N$

Now we write Schr\"odingers eqution in discretized form as
\begin{gather}
-\frac{u_{i+1} -2u_i +u_{i-1}}{h^2}+\rho_i^2u_i=\lambda u_i
\end{gather}
where $u_{i\pm1} = u(\rho_i \pm h)$.

We use that $\rho_i^2=V_i$, so that 
\begin{gather}
-\frac{u_{i+1} -2u_i +u_{i-1}}{h^2}+V_iu_i=\lambda u_i
\end{gather}

Given this eqution we see that it has a shape that can be represented as an matrix equation with an equation on tridiagonal form.
\begin{gather}
A\mathbf{u}=\lambda\mathbf{u}
\label{eigequ}
\end{gather}

where
\begin{gather}
\mathbf{A} = 
\begin{pmatrix}
\frac{2}{\hbar^2} + V_1& -\frac{1}{\hbar^2} & 0 & \dots & 0\\
-\frac{1}{\hbar^2} & \frac{2}{\hbar^2} + V_2& -\frac{1}{\hbar^2} & \ddots & \vdots \\
0 & \ddots & \ddots & \ddots & 0\\
\vdots & \ddots & -\frac{1}{\hbar^2} & \frac{2}{\hbar^2} + V_{N-2} & -\frac{1}{\hbar^2}\\
0 & \dots & 0 & -\frac{1}{\hbar^2} &\frac{2}{\hbar^2} + V_{N-1}
\end{pmatrix},
\quad
\mathbf{u} =
\begin{pmatrix}
u_1\\
u_2\\
u_3\\
\vdots\\
u_{N-1}
\end{pmatrix}
\end{gather}

We observe that this is an eigenvalue problem which we have to solve. For the endpoints we already know the solutions, which are zero for both. 

We are going to use the methode of Jacobi rotation to solve Eq. \eqref{eigequ}. First we have that for a real and symmetric matrix, $\mathbf{A} \in \mathbb{R}^{n\times n}$, there exists a real orthogonal matrix $\mathbf{S}$ such that
\begin{gather}
\mathbf{S}^T\mathbf{AS} = \text{diag}(\lambda_1, \lambda_2, \dots, \lambda_n)
\end{gather}
\clearpage
\section{Results and Discussion}
\clearpage

\section{Conclusion}


\clearpage
\appendix
\section{Appendix}



\end{document}